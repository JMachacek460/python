\documentclass[12pt]{{article}}

%\usepackage[margin=1cm]{{geometry}}  % menší okraje
\usepackage[left=5cm, right=1cm, top=1cm, bottom=1cm]{{geometry}}
\usepackage{{fontspec}}
\usepackage{{stackengine}}
\usepackage{{xcolor}}  % pro práci s barvami
\usepackage{{nopageno}}  % zruší číslování stránek
\stackMath

\newfontfamily\charisfont{{CharisSIL-Regular.ttf}}
\newfontfamily\tengwarfont[Renderer=Graphite]{{tengtelc.ttf}}

% Definice maker musí být před začátkem dokumentu!
\newcommand{{\overlaychars}}[3]{{%
  \hbox{{#1\kern#2#3}}%
}}

\begin{{document}}

% --- STRÁNKA 1: Unicode text ---
\charisfont
\noindent
%Text v běžném Unicode fontu (Charis SIL):

\bigskip

{upraveny_text}

%{upraveny_text}

\bigskip \bigskip
\newpage

% --- STRÁNKA 2: Tengwar s Graphite + správné zalamování ---
\tengwarfont
\sloppy  % uvolní pravidla zalamování
\noindent

%Text prevedeny na Tengwar:

\bigskip
\raggedright %zarovnání k levému okraji

%\Huge  % zvětšené písmo
%\Large
{latex_rendered}


\end{{document}}