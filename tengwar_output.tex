\documentclass[12pt]{article}

%\usepackage[margin=1cm]{geometry}  % menší okraje
\usepackage[left=1cm, right=1cm, top=1cm, bottom=1cm]{geometry}
\usepackage{fontspec}
\usepackage{stackengine}
\usepackage{xcolor}  % pro práci s barvami
\usepackage{nopageno}  % zruší číslování stránek
\stackMath

\newfontfamily\charisfont{CharisSIL-Regular.ttf}
\newfontfamily\tengwarfont[Renderer=Graphite]{tengtelc.ttf}

% Definice maker musí být před začátkem dokumentu!
\newcommand{\overlaychars}[3]{%
  \hbox{#1\kern#2#3}%
}

\begin{document}

% --- STRÁNKA 1: Unicode text ---
\charisfont
\noindent
%Text v běžném Unicode fontu (Charis SIL):

\bigskip

​Ahoj jak se máš!

%​Ahoj jak se máš!

\bigskip \bigskip
\newpage

% --- STRÁNKA 2: Tengwar s Graphite + správné zalamování ---
\tengwarfont
\sloppy  % uvolní pravidla zalamování
\noindent

%Text prevedeny na Tengwar:

\bigskip
\raggedright %zarovnání k levému okraji

%\Huge  % zvětšené písmo
%\Large
\Large\char"200B{\color{red}\char"E02E\char"E040\char"E051}\char"E00F\char"E04A\char"E017\char"200B \mbox{ } \char"200B\char"E017\char"E040\char"E003\char"200B \mbox{ } \char"200B\char"E025\char"E046\char"200B \mbox{ } \char"200B\char"E011\char"E040\char"E040\char"E00A\char"E065


\end{document}